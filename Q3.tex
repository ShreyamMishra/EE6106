\section{Q3}
\label{s:Q3}
\begin{enumerate}[label=(\alph*)]
\item 
The expression for regret is given as:
\begin{equation}
\label{s:eq1}
\bar{R}_n(A) \Let \underset{\bm{y}}{\textrm{sup }}\mathbb{E}[\sum_{t = 1}^{n}y_{A_t, t} - \sum_{t = 1}^{n}\bigl( \underset{1 \leq i \leq K}{\textrm{min}} y_{i,t} \bigr)]
\end{equation}

The key difference is the minimization that occurs inside the summation, i.e prior to taking the sum $y_{i,t}$. This means that the regret is not with respect to the \emph{best expert} in hindsight, but rather the \emph{best expert} in every turn/instance of predicting a bit.

Clearly:
$$
\sum_{t = 1}^{n}\bigl( \underset{1 \leq i \leq K}{\textrm{min}} y_{i,t} \bigr) \leq \bigl( \underset{1 \leq i \leq K}{\textrm{min}}  \sum_{t = 1}^{n}y_{i,t} \bigr)$$
$$\implies \bar{R}_n(A) \geq R_n(A)$$

\item A qualitative argument is as follows: \\

\eqref{s:eq1} is based on a supremum of the table of costs, $\bm{y}$. Clearly $\bar{R}_n(A)$ is maximized if the sum, $ \sum_{t = 1}^{n}\bigl( \underset{1 \leq i \leq K}{\textrm{min}} y_{i,t} \bigr)$ is minimized. Since, $y_{i,t} \geq 0$  $\forall$ $t$, the supremum will occur when the table of costs is designed such that $\forall$ $t$, $\exists$ $i*$ such that $y_{i*,t} = 0$. Furthermore, $\bar{R}_n(A)$ will be maximized if the loss of every other expert's action is maximized, that is:
\[
    y_{A_t,t} = 
\begin{cases}
    0,& \textrm{if} \quad A_t = i*\\
    1,& else \quad 
\end{cases}
\] 

Note, the best expert for a turn/instance $t$ can be randomly designed. In such a scenario, the best algorithm would be randomly, uniformly pick any expert $i \in [k]$ (using arguments similar to \ref{s:Q1}). Thus, in such an event, the expected loss per turn is: $1-\frac{1}{k}$ (since $k-1$ experts give loss 1, 1 expert gives loss 0 and the experts are picked randomly, uniformly).

Thus, in such a case, the regret over the horizon is: $\bar{R}_n(A) = n\bigl(1 - \frac{1}{k}\bigr)$. This is the lower-bound, given the best case strategy. Hence, in general:
$$
\bar{R}_n(A) \geq n\bigl(1 - \frac{1}{k}\bigr)
$$
holds true.

\end{enumerate}
