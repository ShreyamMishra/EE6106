
\documentclass[a4paper,12pt]{article} % 
\usepackage[top = 2.5cm, bottom = 2.5cm, left = 2.5cm, right = 2.5cm]{geometry} 
\usepackage{defs}
\usepackage[T1]{fontenc}
\usepackage[utf8]{inputenc}
\usepackage{graphicx} 
\usepackage{mathtools}
\usepackage{tikz,lipsum,lmodern}
\usepackage[most]{tcolorbox}
\usepackage{setspace}
\setlength{\parindent}{0in}
\usepackage{float}
\usepackage{fancyhdr}
\usepackage{lastpage}
% Installing necessary math packages
\usepackage{amsmath}
\usepackage{amssymb}
\usepackage{latexsym}
\usepackage{enumitem}
\usepackage{bm}
%%%%%%%%%%%%%%%%%%%%%%%%%%%%%%%%%%%%%%%%%%%%%%%%
% 3. Header (and Footer)
%%%%%%%%%%%%%%%%%%%%%%%%%%%%%%%%%%%%%%%%%%%%%%%%

% To make our document nice we want a header and number the pages in the footer.
\usepackage{mathtools}
\pagestyle{fancy} 
\fancyhf{}

\lhead{Assignment-1}% \lhead puts text in the top left corner. \footnotesize sets our font to a smaller size.

%\rhead works just like \lhead (you can also use \chead)
\rhead{Shreyam Mishra, 19D110020} %<---- Fill in your details.
% Similar commands work for the footer (\lfoot, \cfoot and \rfoot).
% We want to put our page number in the center.
\rfoot{Page \thepage \hspace{1pt} of \pageref{LastPage}}
\cfoot{\footnotesize \thepage} 


%%%%%%%%%%%%%%%%%%%%%%%%%%%%%%%%%%%%%%%%%%%%%%%%
% 4. Your document
%%%%%%%%%%%%%%%%%%%%%%%%%%%%%%%%%%%%%%%%%%%%%%%%

% Now, you need to tell LaTeX where your document starts. We do this with the \begin{document} command.
% Like brackets every \begin{} command needs a corresponding \end{} command. We come back to this later.

\begin{document}

%%%%%%%%%%%%%%%%%%%%%%%%%%%%%%%%%%%%%%%%%%%%%%%%
%%%%%%%%%%%%%%%%%%%%%%%%%%%%%%%%%%%%%%%%%%%%%%%%

%%%%%%%%%%%%%%%%%%%%%%%%%%%%%%%%%%%%%%%%%%%%%%%%
% Title section of the document
%%%%%%%%%%%%%%%%%%%%%%%%%%%%%%%%%%%%%%%%%%%%%%%%

% For the title section we want to reproduce the title section of the Problem Set and add your names.

\thispagestyle{empty} % This command disables the header on the first page. 


\begin{center} % Everything within the center environment is centered.
	{\Large \bf EE 6106 - Online Learning and Optimization}
	\vspace{2mm} % to add some vertical space in between the lines
	
	\bf{Assignment-1} % <---- Don't forget to put in the right number
	
	\vspace{2mm}
	
        % YOUR NAMES GO HERE
	{\bf Shreyam Mishra, 19D110020} % <---- Fill in your name and ID number here.
		
\end{center}  

\vspace*{0.3cm} 
%%%%%%%%%%%%%%%%%%%%%%%%%%%%%%%%%%%%%%%%%%%%%%%%
%%%%%%%%%%%%%%%%%%%%%%%%%%%%%%%%%%%%%%%%%%%%%%%%

% Up until this point you only have to make minor changes for every assignment (Number of the assignment). Your write up essentially starts here.

%\textbf{Question 2:} \emph{Prove an analogue of Theorem 1.8 for $n$ functions}. \\
%\textbf{Solution:} LSwT
%
%Analogue of Theorem 1.8 is stated as follows: 
%\begin{tcolorbox}[enhanced,attach boxed title to top center={yshift=-3mm,yshifttext=-1mm},
%  colback=blue!5!white,colframe=blue!75!black,colbacktitle=red!80!black,
%  title=Theorem 1.8's Analogue,fonttitle=\bfseries,
%  boxed title style={size=small,colframe=red!50!black} ]
%Let $u_1, u_2, \hdots , u_n$  be real measurable functions on a measurable space $X$, let $f$ be a continuous mapping of $\mathbb{R}^{n}$ into a topological space $Y$, and define:
%\begin{equation*}
%    h(x) = f(u_1(x), u_2(x), \hdots , u_n(x))
%\end{equation*}
%for $x \in X$. Then $h: X \rightarrow Y$ is measurable.
%\end{tcolorbox}
\section{Q1}
\label{s:Q1}

Consider that the algorithm protocol is as follows:

\[
    A_t = 
\begin{cases}
    0,& w.p \quad p\\
    1,& w.p \quad 1-p
\end{cases}
\]

based on which the environment (adversary) follows the protocol:

\[
    E_t = 
\begin{cases}
    0,& w.p \quad q(p)\\
    1,& w.p \quad 1-q(p)
\end{cases}
\]

where $q(p)$ indicates that the adversary picks its protocol with complete knowledge of our algorithm (and thus the choice of $p$).

Then, the \emph{expected loss} in every turn, denoted ass $\mathbb{E}(\ell)$ is given by:

$$\mathbb{E}(\ell) = q(1-p) + (1-p)q,$$
$$\implies \mathbb{E}(\ell) = p + q - 2pq, $$
$$\implies \mathbb{E}(\ell) = p + q + \frac{1}{2}((p-q)^{2} - (p+q)^{2}) \geq  p + q + \frac{-1}{2}((p+q)^{2}),$$

where the equality holds for $p = q$. Constructing a new variable $[0, 2] \ni t \Let p+q$ and maximizing: $t - \frac{t^2}{2}$ yields $t = 1 \implies p+q = 1$.

Thus, 
$$\mathbb{E}(\ell) \geq 0.5,$$

where the equality holds when $p=q$ and $p+q=1$. Using these two results, we get: $p = 0.5 = q$. In this scenario, the loss per turn is minimized and the reward for the adversary is maximized (any other $q$ for $p=0.5$ yields lesser reward for adversary). 

Hence, over horizon $T$, $\mathbb{E}_{T}(\ell) = \frac{T}{2}$.





\section{Q2}
\label{s:Q2}
Note: This algorithm has been devised as an update to the \emph{majority algorithm} and not the \emph{weighted majority algorithm} in which case the performance can be improved.

The proposed algorithm is as follows:


\begin{itemize}[label=$\circ$, leftmargin = *]
\item Perform the majority algorithm $m$ times. In these trials, the expected loss $L_T = log_{2}K + 1$.
\item At the end of these trials, every expert has made $m$ mistakes (since a trial ends only when every expert has made a mistake). 
\item Thus, the $m+1$ trial can be considered equivalent to the setting with a perfect expert. Hence, the algorithm loss $L_T = O(log_{2}K)$
\end{itemize}
Thus, total loss is $L = (m+1)log_{2}K + m$.
\section{Q3}
\label{s:Q3}
\begin{enumerate}[label=(\alph*)]
\item 
The expression for regret is given as:
\begin{equation}
\label{s:eq1}
\bar{R}_n(A) \Let \underset{\bm{y}}{\textrm{sup }}\mathbb{E}[\sum_{t = 1}^{n}y_{A_t, t} - \sum_{t = 1}^{n}\bigl( \underset{1 \leq i \leq K}{\textrm{min}} y_{i,t} \bigr)]
\end{equation}

The key difference is the minimization that occurs inside the summation, i.e prior to taking the sum $y_{i,t}$. This means that the regret is not with respect to the \emph{best expert} in hindsight, but rather the \emph{best expert} in every turn/instance of predicting a bit.

Clearly:
$$
\sum_{t = 1}^{n}\bigl( \underset{1 \leq i \leq K}{\textrm{min}} y_{i,t} \bigr) \leq \bigl( \underset{1 \leq i \leq K}{\textrm{min}}  \sum_{t = 1}^{n}y_{i,t} \bigr)$$
$$\implies \bar{R}_n(A) \geq R_n(A)$$

\item A qualitative argument is as follows: \\

\eqref{s:eq1} is based on a supremum of the table of costs, $\bm{y}$. Clearly $\bar{R}_n(A)$ is maximized if the sum, $ \sum_{t = 1}^{n}\bigl( \underset{1 \leq i \leq K}{\textrm{min}} y_{i,t} \bigr)$ is minimized. Since, $y_{i,t} \geq 0$  $\forall$ $t$, the supremum will occur when the table of costs is designed such that $\forall$ $t$, $\exists$ $i*$ such that $y_{i*,t} = 0$. Furthermore, $\bar{R}_n(A)$ will be maximized if the loss of every other expert's action is maximized, that is:
\[
    y_{A_t,t} = 
\begin{cases}
    0,& \textrm{if} \quad A_t = i*\\
    1,& else \quad 
\end{cases}
\] 

Note, the best expert for a turn/instance $t$ can be randomly designed. In such a scenario, the best algorithm would be randomly, uniformly pick any expert $i \in [k]$ (using arguments similar to \ref{s:Q1}). Thus, in such an event, the expected loss per turn is: $1-\frac{1}{k}$ (since $k-1$ experts give loss 1, 1 expert gives loss 0 and the experts are picked randomly, uniformly).

Thus, in such a case, the regret over the horizon is: $\bar{R}_n(A) = n\bigl(1 - \frac{1}{k}\bigr)$. This is the lower-bound, given the best case strategy. Hence, in general:
$$
\bar{R}_n(A) \geq n\bigl(1 - \frac{1}{k}\bigr)
$$
holds true.

\end{enumerate}

\section{Q4}
\label{s:Q4}
$$\mathbb{P}(S_n = k) = \Mycomb{k}p^{k}(1-p)^{k}.$$
Next, 
$$S_n > na \implies e^{sS_n} > e^{sna} \quad (s > 0),$$
$$\implies \mathbb{P}(S_n > na) = \mathbb{P}(e^{sS_n} > e^{sna}) \leq \mathbb{E}[e^{sS_n}]e^{-san} \quad \text{(Chernoff bounding)}$$

Computing the expression on right-hand side:
$$\mathbb{E}[e^{sS_n}]e^{-san} = e^{-san}\sum_{k = 0}^{n}\Mycomb{k}p^{k}(1-p)^{k}e^{sk} = e^{-san}((1-p) + pe^{s})^{n},$$
$$
\implies \mathbb{E}[e^{sS_n}]e^{-san} = e^{-san}\sum_{k = 0}^{n}\Mycomb{k}p^{k}(1-p)^{k}e^{sk} = e^{-san + nlog(pe^s + 1-p)}.
$$

Defining, $f(s) \Let -san + nlog(pe^s + 1-p)$, we minimize by differentiating the expression with respect to s, To place the tightest bounds:

$$\frac{d}{ds}(f(s)) = -an + \frac{npe^s}{pe^s + 1- p} = 0$$
$$\implies s^{*} = log(\frac{(1-p)a}{(1-a)p})$$
$$\implies g(s^{*}) = n(log(\frac{(1-p)}{(1-a)} - alog(\frac{(1-p)a}{(1-a)p}) = -n(alog(\frac{a}{p}) + (1-a)log(\frac{(1-a)}{(1-p)})$$
$$\implies \mathbb{P}(S_n \geq na) \leq e^{-g(s^{*})}$$
\section{Q5}
\label{s:Q5}
$$X \geq x \implies e^{sX} \geq e^{sx} (s >0)$$
$$\implies \mathbb{P}(X \geq x) = \mathbb{P}(e^{sX} \geq e^{sx}) \leq e^{-sx}\mathbb{E}[e^{sX}]$$
Defining the right-hand side as follows:
$$\underset{s}{\text{min}} f(s) \Let e^{-sx}\mathbb{E}[e^{sX}] = e^{-sx}\int_{-\infty}^{\infty}e^{sy}\frac{1}{\sqrt{2\pi}}e^{\frac{-y^2}{2}}dy$$

Computing the definite integral yields:
$$\underset{s}{\text{min}} f(s) \Let e^{-sx}\mathbb{E}[e^{sX}] = e^{-sx + \frac{s^2}{2}}$$

The minimum occurs at $s=x \implies \mathbb{P}(X \geq x) = \mathbb{P}(e^{sX} \geq e^{sx}) \leq e^{\frac{-x^2}{2}}.$


\section{Q6}
\label{s:Q6}
By definition of a sub-Gaussian random variable, we may claim:
$$\mathbb{E}[e^{t(X-\mu)}] \leq e^{\frac{t^2\sigma^{2}}{2}}, \forall t \in \mathbb{R}$$

Taylor expansion of $e^{t(X-\mu)} = 1 + t(X-\mu) + \frac{t^2}{2}((X-\mu))^{2} + o(t^2)$. Using linearity of expectation operator, the right hand side of the inequality is given by:
$$\mathbb{E}[e^{t(X-\mu)}] = 1 + t\mathbb{E}[(X-\mu)] + \frac{t^2}{2}\mathbb{E}[((X-\mu))^2] + o(t^2).$$

The Taylor expansion of left hand side of the inequality is given by:
$$e^{\frac{t^2\sigma^{2}}{2}} = 1 + \frac{t^2\sigma^{2}}{2} + o(t^2)$$

Using $\mathbb{E}[(X-\mu)] = $, and comparing the Taylor expansion of both sides of the inequality, we get:
$$\frac{t^2}{2}\mathbb{E}[((X-\mu))^2] leq \frac{t^2\sigma^{2}}{2} \implies Var[X] \leq \sigma^2$$
 
\end{document}
